\documentclass[a4paper]{article}

\usepackage[utf8]{inputenc}
\usepackage[T1]{fontenc}
\usepackage[spanish]{babel}
\usepackage{amsmath, amssymb}
\usepackage{graphicx}
\usepackage{geometry}

\geometry{a4paper, margin=1in}

\title{Programación de robots - cinemática - apuntes de classe}
\author{Carlos Pérez Aranda}
\date{\today}

\begin{document}

\maketitle

\begin{abstract}
En este documento se presentan los conceptos fundamentales de la cinemática aplicada a la programación de robots móviles.
Se abordan las definiciones de grados de libertad desde perspectivas matemáticas e ingenieriles, 
así como la clasificación de los robots en función de su holonomía. Además, se introduce un modelo 
cinemático básico de velocidad constante, destacando sus limitaciones y aplicaciones prácticas. Estos apuntes 
están orientados a proporcionar una base teórica para el diseño y análisis de sistemas robóticos móviles.
\end{abstract}

\section{Introducción}
El modelo puede seer Sencillo o complicado.

El primero que vamos a ver es muy sencillo pero tidene problemas.
El mundo real tiene inexactitudes, como el radio de la rueda, el peso del robot, la fricción, etc. En la realidad el robot
puede no estar en su mejor momento, baches en el suelo, eso va a introducir errores en la cinemática.
Los grados de libertad es cuantos motoress y articulaciones tenemos.
2 definiciones matem-aticas, mínimo número de valores independientes definen la posición de un objeto en el espacio. Coche X,Y, rotación.
Cuadricóptero, 3 ejes de rotación y 3 de traslación. 6 grados de libertad.

Definición "ingenieril": número de motores y articulaciones que tiene el robot.
Coche, 2 grados de libertad, 2 motores, 2 ruedas. Un motor para cada rueda.
Cuadricóptero, 4 grados de libertad.

Dependiendo de la relación entre los grados de libertad matemáticos A e ingenieriles B.
\begin{itemize}
    \item $A = B$: vehículo holónomo.
    \item $A > B$: vehículo no holónomo.
    \item $A < B$: vehículo no redundante.
\end{itemize}

Robot con ruedas holónomo lleva ruedas suecas, que pueden girar en cualquier dirección. BB-8 es holónomo.



\section{Modelos cinemáticos para robots móviles}
\subsection{Modelo cinemático de velocidad constante}
Muy simple, puede aplicarse sin sensor de odometría.

No se adapta a caminos con cambios abruptos de dirección, como un giro brusco.

Escribe aquí el desarrollo del contenido.

vk = vk-1.

El prototipo de la función será el siguiente:

function [newPose] = constantSpeedKinematics(xinit,v,deltat)

Es un modelo no homolónomo.

function [derPose] = diffDriveKinematics(tita,ur,ul,L,r)

\subsection{Modelo diferencial}
Se usa en los legos entre otros.
El estado de robot corresponde a su pose y el vector de acción es las velocidaddes angulares de las ruedas.

\subsection{Modelo de la bicicleta}
Estamos suponiendo que las ruedas son rígidas y tienen solo un punto de contacto con el suelo.
Hay rozamiento para evitar deslizamientos.
Rueda trasera fija, rueda delantera libre.
No desliza lateralmente.
Tenemos dos ángulos, la orientación de la bicicleta y la orientación de la rueda delantera.

Si la vel lineal es 0 no hay cambio de orientación.
Si el ángulo de la rueda es pi medios entramos en una indefinición.

function [derPose] = bikeKinematics(tita,v,L,gamma)

\section{Conclusión}
Escribe aquí la conclusión.

\end{document}